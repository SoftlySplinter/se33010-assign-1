\documentclass[a4paper, 10pt, notitlepage]{article}

\usepackage{fullpage}
%\usepackage[cm]{fullpage}
\usepackage{glossaries}

\title{SE33010 Assignment One}
\author{Alexander D Brown (adb9)}

\newacronym{vdm}{VDM}{Vienna Development Model}
\newacronym{z}{Z}{Z Notation}

\begin{document}

\begin{centering}
\section*{SE33010 Assignment One - Alexander D Brown (adb9)}
\subsection*{Comparing Vienna Development Model and Z Notation}
\end{centering}

Both the \gls{vdm} and \gls{z} use a model-base specification technique and share a lot of their 
mathematical notation. Where they differ is in the way their specifications are written.

Both are concentrated on the specification of \textit{abstract machines}; a ``model orientated'' 
approach. This approach differs from an ``algebraic'' (or ``property orientated'') approaches
which focus on defining \textit{abstract data types}\cite{Hayes93vdmz}.

\bibliographystyle{plain}
\bibliography{adb9_se33010_assignment_1}


\end{document}

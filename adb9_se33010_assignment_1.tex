\documentclass[a4paper, 10pt, notitlepage, fleqn]{article}

\usepackage{fullpage}
%\usepackage[cm]{fullpage}
\usepackage{glossaries}
\usepackage{algorithmicx}
\usepackage{algpseudocode}
\usepackage{amsmath}
\usepackage{mathtools}

\title{SE33010 Assignment One}
\author{Alexander D Brown (adb9)}

\newacronym{vdm}{VDM}{Vienna Development Model}
\newacronym{z}{Z}{Z Notation}

\begin{document}

\begin{centering}
\section*{SE33010 Assignment One - Alexander D Brown (adb9)}
\subsection*{Comparing Vienna Development Model and Z Notation}
\end{centering}

Both the \gls{vdm} and \gls{z} use a model-base specification technique and share a lot of their 
mathematical notation. Where they differ is in the way their specifications are written.

Both are concentrated on the specification of \textit{abstract machines}; a ``model orientated'' 
approach. This approach differs from an ``algebraic'' (or ``property orientated'') approaches
which focus on defining \textit{abstract data types}\cite{Hayes93vdmz}.

Algebraic approaches give no explicit model of type, defining abstract data types in terms of the 
relationships between its interactions. In contrast, both \gls{vdm} and \gls{z} give an explicit 
model of the state of an abstract machine. A commonly used example is a stack; in an algebraic 
approach it might be defined in a manner described in equation~\ref{eq:algebraic-stack}, whilst
a model orientated approach would typically be modelled as a sequence.

\begin{equation}\label{eq:algebraic-stack}
\mathbf{pop}(\mathbf{push} (x,s)) = s
\end{equation}

Where \gls{vdm} and \gls{z} start to differ is in the definitions; \gls{vdm} has a lot of 
structure, using keywords to define different parts of a specification. \Gls{z} has very little of
this structure; the definitions are done during the specification and usually consist of
\textit{schemas}. These schema define not only the states of an abstract machine, but also the 
operations.

\Gls{z} provides a lot more in-built binary relations, this can lead to simpler predicates in
specifications as this tends to map to real life data types. However they can also be modelled in
a similar way to that of \gls{vdm}. These \gls{z} relations also offer a rich set of operators
defined upon them.

Operations yield some of the biggest differences between \gls{vdm} and \gls{z}; syntactically
there are quite a few differences; \gls{vdm} users hooked variables 
($\overset{\leftharpoonup}{a}$) to define variables in the \textit{before} and unhooked ($a$) for
variables in the \textit{after} state.

\Gls{z}, in contrast, uses undecorated variables ($a$) for variables in the \textit{before} state,
and primed variables ($a'$) for the \textit{after} state. \Gls{z} also uses variables ending in 
`$?$' for input and variables ending in `$!$' for outputs.

\Gls{vdm} explicitly defines the `pre-' and `post-' conditions. It also uses an externals clause
to define the access to a variable. \Gls{z} has no equivalent to this clause, the predicate has to
explicitly define all final values of variables, even if that variable is unchanged, leading to
statements like: $a' = a$ at the end of predicates.



\bibliographystyle{plain}
\bibliography{adb9_se33010_assignment_1}


\end{document}
